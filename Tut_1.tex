\documentclass{beamer}
\usepackage[ngerman]{babel}
\usepackage[utf8]{inputenc}
\usepackage{graphicx}
\usepackage{amssymb} 

\usetheme{Warsaw}
\usecolortheme{default}


\author{Tristan Schnell}
\title{GBI-Tutorium 1}
\date{27. Oktober 2011}


\begin{document}

\begin{frame}
	\titlepage
\end{frame}

\begin{frame}
	\frametitle{Inhaltsverzeichnis}
	\tableofcontents
\end{frame}

\section{Organisatorisches}

\begin {frame}

	\begin{block}{Vorstellung}
		\begin{columns}
       			 \column{0.8\textwidth}
               			\begin{itemize}
					\item Name
					\item Alter
					\item Studiengang (Semester)
					\item Woher?
					\item Erwartungen
				\end{itemize}
		\end{columns}
	\end{block}
\end {frame}

\begin {frame}
	\frametitle {Organisatorisches}
	\begin{block}{Tutorium ist}
        		\begin{itemize}
			\item kurze Wiederholung der Vorlesung
			\item Anlaufstelle für Fragen
			\item Übungsbereich für aktuellem Vorlesungsstoff
			\item Ausgabestelle der Übungsblätter
			\item Freiwillig
		\end{itemize}
	\end{block}
	\pause
	\begin{block}{Tutorium ist nicht}
        		\begin{itemize}
			\item Vorlesungsersatz
			\item Lösungsstelle für kommendes Übungsblatt
		\end{itemize}
	\end{block}
\end {frame}

\begin {frame}
	\frametitle {Organisatorisches}
	\begin{block}{Übungsblatt}
        		\begin{itemize}
			\item Übungsblatt einzelnd handschriftlich bearbeiten
			\item Abgabe Freitag 12:30 Uhr im Briefkasten im Keller
			\item Offensichtlich abgeschrieben $\Rightarrow$ 0 Punkte
			\item Ab Hälfte der Punkte bestanden (Voraussichtlich 120)
			\item Übungsschein zum Bestehen des Moduls notwendig
		\end{itemize}
	\end{block}
\end {frame}

\begin{frame}

	\begin{block}{Klausur}
        		\begin{itemize}
			\item 5. März 2012 (11 Uhr)
			\item Nachprüfung im September
			\item Klausur ist Teil der Orientierungsprüfung.
		\end{itemize}
	\end{block}
	\pause
	\frametitle {Organisatorisches}
	\begin{block}{Kontakt / Information}
		\begin{itemize}
			\item gbi.tutorium@googlemail.com
			\item http://gbi.ira.uka.de/
		\end{itemize}
	\end{block}
\end{frame}

\section{Alphabete}

\begin{frame}
	\frametitle{Alphabete}
	\begin{block}{Definition}
		Ein Alphabet ist eine \emph{endliche, nichtleere} Menge von Zeichen.
	\end{block}
	\begin{exampleblock}{Aufgaben}
		\begin{itemize}
			\item $\mathbb N$\(_{+}\) ?
			\item \(M=\{\phi,3,\psi,a\}\) ?
		\end{itemize}
	\end{exampleblock}
	\pause
	\begin{alertblock}{Notation}
		\begin{itemize}
			\item $\mathbb N$\(_{+}=\{1,2,3,\dots\}\) (positive ganze Zahlen)
			\item $\mathbb N$\(_{0}=\{0,1,2,3,\dots\}\) (nichtnegative ganze Zahlen)
		\end{itemize}
	\end{alertblock}
\end{frame}

\section{Aussagenlogik}

\begin {frame} 
	\frametitle {Aussagenlogik}
	\begin{block}{}
		\begin{itemize}
			\item Eine Aussage ist ein Satz, der (objektiv) entweder wahr oder falsch sein kann
			\item Aussagen sind äquivalent (\(\Leftrightarrow\)), wenn sie die gleichen Wahrheitswerte besitzen
		\end{itemize}
	\end{block}
	\pause
	\begin{block}{Logisches UND und ODER}
	\begin{center}
		\begin{tabular}{|c|c||c| }
			\hline
			A 		&  B 		& A $\land$ B	\\
			\hline
			wahr 		& wahr 	& \textbf{wahr}	\\
			\hline
			wahr 		& falsch 	& falsch	\\
			\hline
			falsch 		& wahr 	& falsch	\\
			\hline
			falsch 		& falsch 	& falsch	\\
			\hline
		\end{tabular}
		\begin{tabular}{|c|c||c| }
			\hline
			A 		&  B 		& A $\lor$ B	\\
			\hline
			wahr 		& wahr 	& wahr	\\
			\hline
			wahr 		& falsch 	& wahr	\\
			\hline
			falsch 		& wahr 	& wahr	\\
			\hline
			falsch 		& falsch 	& \textbf{falsch}	\\
			\hline
		\end{tabular}
	\end{center}
	\end{block}
	\pause	
	\begin{exampleblock}{Aufgabe}
		Stelle eine Wahrheitstabelle für den Ausdruck \((A\wedge B)\vee A\) auf.
	\end{exampleblock}
\end{frame}

\begin {frame} 
	\frametitle {Implikation}
	\begin{center}
		\begin{tabular}{|c|c||c| }
			\hline
			A 		&  B 		& $\Rightarrow$ 	\\
			\hline
			wahr 		& wahr 	& wahr	\\
			\hline
			wahr 		& falsch 	& \textbf{falsch}	\\
			\hline
			falsch 		& wahr 	& wahr	\\
			\hline
			falsch 		& falsch 	& wahr	\\
			\hline
		\end{tabular}
	\end{center}

	\begin{alertblock}{Wichtig!}
		\begin{itemize}
			\item A $\Rightarrow$ B ist äquivalent zu \(\neg A\vee B\)
			\item D.h. man muss nur etwas tun, wenn A wahr ist. (Beweise)
		\end{itemize}
	\end{alertblock}
\end{frame}

\begin {frame} 
	\frametitle {Implikation}
	\begin{center}
		\begin{tabular}{|c|c||c| }
			\hline
			A 		&  B 		& $\Rightarrow$ 	\\
			\hline
			wahr 		& wahr 	& wahr	\\
			\hline
			wahr 		& falsch 	& \textbf{falsch}	\\
			\hline
			falsch 		& wahr 	& wahr	\\
			\hline
			falsch 		& falsch 	& wahr	\\
			\hline
		\end{tabular}
	\end{center}
	\begin{exampleblock}{Aufgabe}
		Finde für F einen äquivalenten Ausdruck, in dem A und B jeweils höchstens einmal vorkommen.
		\begin{displaymath}
			F = (A\Rightarrow B) \Rightarrow ((B\Rightarrow A) \Rightarrow B)
		\end{displaymath}
	\end{exampleblock}
\end{frame}

\section{Relationen}

\subsection{Kartesisches Produkt}
\begin {frame}
	\frametitle {Kartesisches Produkt}
	\begin{block}{Definition}
		\(A\times B = \{(a,b)|a\in A \wedge b\in B\}\)\\
		Die Menge aller geordneten Paare (a,b) mit a aus A und b aus B
	\end{block}
	\pause
	\begin{exampleblock}{Aufgaben}
		\begin{itemize}
			\item Berechne \(\{a,b\} \times \{1,2,3\}\).\\
			\uncover<2->{ \(\{(a,1),(a,2),(a,3),(b,1),(b,2),(b,3)\}\) }
			\item Wieviele Elemente hat \(\{\alpha,\beta,\gamma,\delta\} \times \{42,43,44\}\)?\\
			\uncover<3->{12}
			\item Was ist \(\emptyset \times M\)?\\
			\uncover<4->{\(\emptyset\)}
		\end{itemize}
	\end{exampleblock}
\end {frame}

\subsection{Relationen}
\begin{frame}
	\frametitle{Relationen}
	\begin{block}{Definition}
		\begin{itemize}
			\item Eine Teilmenge \(R\subseteq A \times B\) heißt (binäre) Relation von A in B.
			\item Wenn A = B, spricht man von einer Relation auf der Menge A.
			\item Statt (a,b) $\in$ R kann man auch a R b schreiben bzw. statt \((a,b) \in R_{\geq}\) auch \(a \geq b\).
		\end{itemize}
	\end{block}
	\pause
	\begin{exampleblock}{Aufgabe}
		Wie ist die Kleiner-Gleich-Relation \(R_{\leq}\) auf der Menge M = \{1,2,3\} formell definiert?\\
	\pause
		\uncover<2->{\(R_{\leq}=\{(1,1),(1,2),(1,3),(2,2),(2,3),(3,3)\}\)}
	\end{exampleblock}
\end{frame}

\begin {frame}
	\frametitle {Eigenschaften von Relationen}
	\begin {tabular} {l l}
		\textbf {linkstotal} 			& eine Relation R $ \subseteq $ A x B ist linkstotal wenn gilt: \\
							& $ \forall $ a $ \in $ A, $ \exists $ b $ \in $ B : ( a , b ) $ \in $ R \\
		\\
		\textbf {rechtseindeutig} 		&eine Relation R $ \subseteq $ A x B ist rechtseindeutig wenn gilt: \\
							& $ \forall $ a $ \in $ A, $ \forall $ b , c $ \in $ B : \\ 
							& ( a , b ) $ \in $ R $ \land $ ( a , c ) $ \in $ R $\Rightarrow$ b = c \\
		\\
		\textbf {rechtstotal } 		& eine Relation R $ \subseteq $ A x B ist rechtstotal wenn gilt: \\
							& $ \forall $ b $ \in $ B, $ \exists $ a $ \in $ A : ( a , b ) $ \in $ R \\
		\\
		\textbf {linkseindeutig} 		&eine Relation R $ \subseteq $ A x B ist linkseindeutig wenn gilt: \\
							& $ \forall $ a , c $ \in $ A, $ \forall $ b $ \in $ B : \\ 
							& ( a , b ) $ \in $ R $ \land $ ( c , b ) $ \in $ R $\Rightarrow$ a = c \\
	\end {tabular}
\end {frame}

\begin {frame}
	\frametitle {Eigenschaften von Relationen}
	\begin {tabular} {l l}
		\textbf {linkstotal} 			& Jedes Element aus A hat mindestens einen Partner\\  &in B \\
		\\
		\textbf {rechtseindeutig} 		& Jedes Element aus A hat höchstens einen Partner\\ & in B \\ \\
		\\
		\textbf {rechtstotal } 		& Jedes Element aus B hat mindestens einen Partner\\ & in A
		\\
		\\
		\textbf {linkseindeutig} 		& Jedes Element aus B hat höchstens einen Partner\\&  in A 
		\\
	\end {tabular}
\end {frame}	

\begin {frame}
	\frametitle{Eigenschaften von Relationen}
	 \begin{exampleblock}{Aufgaben}
		Sind folgende Relationen links-/rechtstotal, links-/rechtseindeutig?
		\begin{itemize}
			\item Die Gleichheitsrelation \(R_{=}\) auf $\mathbb R$
			\item Die Kleinerrelation \(R_{<}\) auf $\mathbb R$
		\end{itemize}
	\end{exampleblock}
\end{frame}

\subsection{Funktionen/Abbildungen}
\begin {frame}
	\frametitle{Funktionen/Abbildungen}
	\begin{block}{Definition}
		Eine Relation, die linkstotal und rechtseindeutig ist, nennt man Funktion oder Abbildung. \\
		Sei f: A $\rightarrow$ B eine Funktion. Dann ist:
		\begin{itemize}
			\item A der Definitionsbereich
			\item B der Zielbereich
			 \item f(A) der Bildbereich von f
		\end{itemize}
	\end{block}
	\begin{exampleblock}{Aufgabe}
		Was bedeutet es wenn der Bildbereich gleich dem Zielbereich ist?
	\end{exampleblock}
\end {frame}

\begin {frame}
	\begin{block}{Eigenschaften von Funktionen/Abbildungen}
		\begin{itemize}
			\item linkseindeutig $\rightarrow$ injektiv
			\item rechtstotal $\rightarrow$ surjektiv
			\item injektiv + surjektiv = bijektiv
		\end{itemize}
	\end{block}
	\begin{exampleblock}{Aufgaben}
		Sind folgende Funktionen injektiv, surjektiv oder bijektiv?
		\begin{itemize}
			\item \(f:\mathbb R \rightarrow \mathbb R, x \mapsto x\)
			\item \(g:\mathbb N_{0} \rightarrow \mathbb N_{0}, x \mapsto 2x\)
		\end{itemize}
	\end{exampleblock}
\end{frame}

\begin{frame}{Fragen}
	\begin{center}
		Fragen?!
	\end{center}
\end{frame}

\begin{frame}{Fragen}
	\frametitle{Unnützes Wissen}
	\begin{center}
		Anatidaephobia ist die Angst von einer Ente beobachtet zu werden.
	\end{center}
\end{frame}
\end{document}
