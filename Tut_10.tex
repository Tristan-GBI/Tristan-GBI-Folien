\documentclass{beamer}
\usepackage[ngerman]{babel}
\usepackage[utf8]{inputenc}
\usepackage{graphicx}
\usepackage{amssymb} 
\usepackage{amsmath}

\usepackage{hyperref}

\usetheme{Warsaw}
\usecolortheme{default}


\author{Tristan Schnell}
\title{GBI-Tutorium 10}
\date{12.Januar 2012}

\begin{document}

\begin {frame}
	\titlepage
\end {frame}

\begin {frame}
	\frametitle {Inhaltsverzeichnis}
	\tableofcontents
\end {frame}

\section{Master-Theorem}

\begin{frame}
	\frametitle{Quicksort}
	\begin{tabbing}
	\textbf{quick}\=\textbf{sort}\=(links, rechts)\\
     	\>\textbf{if} (links $<$ rechts) \textbf{do}\\
         	\> \>teiler := teile(links, rechts)\\
         	\> \>quicksort(links, teiler-1)\\
         	\> \>quicksort(teiler+1, rechts)\\
     	\>\textbf{od}
	\end{tabbing}
\end{frame}

\begin{frame}
	\frametitle{Master-Theorem}
	\begin{block}{Bietet}
		Einfache Lösung für Gleichungen der Form:\\
		$a\cdot T(n/b) +f(n)$\\
		\pause \bigskip
		Geht leider nicht immer
	\end{block}
\end{frame}


\begin{frame}
	\frametitle{Master-Theorem}
	\begin{block}{Ein paar Aufgaben}
		\begin{itemize}
			\item $T(n) = 9\cdot T(n/3) + n^2 + 2n + 1$
			\item $T(n) = \sqrt{3}\cdot T(n/2) + \log{n}$
			\item $T(n) = 4\cdot T(n/4) + n \log{n}$
			\item $T(n) = 2^n\cdot T(n/2) + n^n$
		\end{itemize}
	\end{block}
\end{frame}

\section{Automaten}

\begin{frame}
	\frametitle{Automaten}
	\begin{block}{Mealy Automaten}
		\begin{itemize}
			\item eine endliche Zustandsmenge Z
			\item einen Anfangszustand $z_0$ $\in$ Z
			\item ein Eingabealphabet X
			\item eine Zustandsüberführungsfunktion f: Z x X $\rightarrow$ Z
			\item ein Ausgabealphabet Y
			\item eine Ausgabefunktion g : Z x X $\rightarrow$ Y$^\ast$
		\end{itemize}	
	\end{block}
\end{frame}

\begin{frame}
	\frametitle{Automaten}
	\begin{block}{Moore Automaten}		
		\begin{itemize}
			\item eine endliche Zustandsmenge Z
			\item einen Anfangszustand $z_0$ $\in$ Z
			\item ein Eingabealphabet X
			\item eine Zustandsüberführungsfunktion f: Z x X $\rightarrow$ Z
			\item ein Ausgabealphabet Y
		\end{itemize} 
	\end{block}
\end{frame}

\begin{frame}
	\frametitle{Automaten}
	\begin{block}{Akzeptoren}
		Akzeptoren haben einen oder mehrere Akzeptierte Zustände, endet der Automat nach der Eingabe an einem dieser Zustände ist das Eingabewort akzeptiert.
		Diese Zustände werden mit einem doppelten Kreis gekennzeichnet und werden akzeptierende Zustände genannt.
		
		Mit einem solchen Akzeptor können Formale Sprachen überprüft werden.
	\end{block}
\end{frame}

\begin {frame}
\frametitle {Ende}
	\begin {center}
		Noch Fragen?
	\end {center}
\end {frame}

\begin {frame}
\frametitle {Unnützes Wissen}
	\begin {center}
		Das Pfeifen unter Wasser ist in Florida verboten.
	\end {center}
\end {frame}

\end {document}