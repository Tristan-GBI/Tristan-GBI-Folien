\documentclass{beamer}
\usepackage[ngerman]{babel}
\usepackage[utf8]{inputenc}
\usepackage{graphicx}
\usepackage{amssymb} 

\usetheme{Warsaw}
\usecolortheme{default}


\author{Tristan Schnell}
\title{GBI-Übungen 1}
\date{27. Oktober 2011}

\begin{document}

\begin{frame}[label=start]
	\frametitle{Aufgaben}
	\begin{block}{Algorithmen}
		\begin{itemize}
			\item \hyperlink{1.1}{Invariante finden+beweisen}
			\item \hyperlink{1.2}{Invariante beweisen/widerlegen}
			\item \hyperlink{1.3}{Algorithmus finden}
		\end{itemize}
	\end{block}
\end{frame}

\begin{frame}[label=1.1]
	\frametitle{Aufgabe 1}
	\begin{block}{Aufgabe}
		
		\begin{tabbing}
			$x \leftarrow a$\\
			$y \leftarrow b$\\
			\textbf{for} \=$i \leftarrow 0$ \textbf{to} $a+b+1$ \textbf{do}\\
			\> $k \leftarrow min(x,y)$\\
			\> $g \leftarrow max(x,y)$\\
			\> $x \leftarrow k$\\
			\> $y \leftarrow g-k$\\
			\textbf{od}\\
			\\
			$a,b \in \mathbb{N}_0$ und $a + b \ge 1$\\
			\\
			Finde und beweise eine aussagekräftige Schleifeninvariante.
		\end{tabbing}
	\end{block}
\end{frame}

\begin{frame}
	\frametitle{Aufgabe 2}
	\begin{block}{Lösung}
		$ggT(a,b) = ggT(x,y)$
	\end{block}
	\begin{block}{Beweis}
		\begin{tabbing}
			\pause
			sei $k = ggT(a,b)$\\
			\pause
			$\Rightarrow$ \= $a = k\cdot n_1$ und $  b = k\cdot n_2$\\
			\pause
			sei oBdA $a \ge b$\\
			\pause
			$\Rightarrow$ Nach erstem Schleifendurchlauf:\\
			\pause
			\>$x=b$ und $ y=a-b$\\
			\pause
			$\Rightarrow x = k\cdot n_2 $ und $  y = (k\cdot n_1 + k\cdot n_2) = k \cdot (n_1+n_2) = k \cdot n_3$\\
			etc\dots
		\end{tabbing}
	\end{block}
	\hyperlink{start}{\beamergotobutton{zurück}}
\end{frame}

\begin{frame}[label=1.2]
	\frametitle{Aufgabe 2}
	\begin{block}{Aufgabe}
		
		\begin{tabbing}
			$X_0 \leftarrow 2$\\
			$Y_0 \leftarrow 5$\\
			\textbf{for} \=$i \leftarrow 0$ \textbf{to} $n$ \textbf{do}\\
			\> $j \leftarrow i$\\
			\> $Y_{j+1} \leftarrow 5Y_j - 6X_j$\\
			\> $X_{j+1} \leftarrow Y_j$\\
			\> $y \leftarrow g-k$\\
			\textbf{od}\\
			\\
			\\
			Beweise oder widerlege:\\
			$Y_j = 2^{j+1}+3^{j+1} \land X_j = 2^j + 3^j$ ist Schleifeninvariante.
		\end{tabbing}
	\end{block}
\end{frame}

\begin{frame}
	\frametitle{Aufgabe 2}
	\begin{block}{Lösung}
		Vollständige Induktion!
	\end{block}
	\pause
	\begin{block}{Induktionsanfang}
		$j=0:$\\
		$2^{0+1} + 3^{0+1} = 5 = Y_0 \land 2^0 + 3^0 = 2 = X_0$

		Rest Tafel
	\end{block}
	\hyperlink{start}{\beamergotobutton{zurück}}
\end{frame}

\begin{frame}[label=1.3]
	\frametitle{Aufgabe 3}
	\begin{block}{Aufgabe}
		Schreiben Sie einen Algorithmus auf, der folgendes leistet:
		\begin{itemize}
			\item Als Eingaben erhält er ein Wort $w : \mathbb{G}_n \rightarrow A$ und zwei 					Symbole $x \in A$ und $y \in A$.
			\item Am Ende soll eine Variable r den Wert 0 oder 1 haben, und zwar soll gelten:\\
				r = $\begin{cases} 1 & $falls irgendwo in $w$ direkt$\\ & $hintereinander erst  				x und dann y vorkommen$\\
				0 & $sonst$\end{cases}$
		\end{itemize}
		Benutzen Sie zum Zugriff auf das $i$-te Symbol von $w$ die Schreibweise $w(i)$.\\
		Formulieren Sie den Algorithmus mit Hilfe einer \textbf{for}-Schleife.
	\end{block}
\end{frame}

\begin{frame}
	\frametitle{Aufgabe 3}
	\begin{block}{Lösung}
		\begin{tabbing}
			$p \leftarrow 0$\\
			\textbf{for} \= $i \leftarrow 0$ \textbf{to} $n-2$ \textbf{do}\\
			\>$p \leftarrow$
			$\begin{cases}$
				1 $&$ falls $w(i)=x \land w(i+1) = y\\$
				p $&$ sonst
			$\end{cases}$
		\end{tabbing}
	\end{block}
	\hyperlink{start}{\beamergotobutton{zurück}}
\end{frame}

\end{document}