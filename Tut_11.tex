\documentclass{beamer}
\usepackage[ngerman]{babel}
\usepackage[utf8]{inputenc}
\usepackage{graphicx}
\usepackage{amssymb} 
\usepackage{amsmath}

\usepackage{hyperref}

\usetheme{Warsaw}
\usecolortheme{default}


\author{Tristan Schnell}
\title{GBI-Tutorium 11}
\date{19.Januar 2012}

\begin{document}

\begin {frame}
	\titlepage
\end {frame}

\begin {frame}
	\frametitle {Inhaltsverzeichnis}
	\tableofcontents
\end {frame}

\section{Reguläre Ausdrücke}

\begin{frame}
	\frametitle{Reguläre Ausdrücke}
	\begin{block}{Definition}
		\begin{itemize}
			\item sei A ein Alphabet, das kein Zeichen aus Z enthält
			\item sei Z das Alphabet $Z = \{ \mid , (,), \ast , \emptyset \}$
			\item regulärer Ausdruck über A ist eine Zeichenfolge über dem Alphabet A 	$\cup $ Z, die folgenden Vorschriften genügt:
			\pause
				\begin{itemize}
					\item $\emptyset$ ist ein regulärer Ausdruck
					\pause
					\item Für jedes x $\in$ A ist x ein regulärer Ausdruck
					\pause
					\item wenn $R_1$ und $R_2$ reguläre Ausdrücke, dann auch $(R_1 \mid R_2),(R_1 R_2)$ und $(R_1 \ast )$
				\end{itemize}
				\pause
			\item die von einem regulären Ausdruck R beschriebene formale Sprache ist $<R>$
		\end{itemize}
	\end{block}
\end{frame}

\begin{frame}
	\frametitle{Reguläre Ausdrücke}
	\begin{block}{Eigenschaften}
		\begin{itemize}
			\item $R \ast \ast = $ \pause $R\ast$
			\item $<R> = \{ \epsilon \} \Rightarrow R =$ \pause $ \emptyset \ast $
			\item Sei $L = <R>$, dann gilt
				\begin{itemize}
					\item $L^{\ast} = $ \pause $<(R) \ast >$
					\item $L^+ = $ \pause $<R(R) \ast > $
				\end{itemize}
		\end{itemize}
	\end{block}
\end{frame}

\begin{frame}
	\frametitle{Reguläre Ausdrücke}
	\begin{block}{Gebe regulären Ausdruck an $(A = \{ a,b \})$}
		\begin{itemize}
			\item Alle Wörter in denen das Teilwort abb vorkommt
			\item Die Sprache aller Wörter, in denen mindestens drei b vorkommen
			\item Die Sprache aller Wörter, in denen nirgends das Teilwort ab vorkommt
			\item $ L = \{ a^n b^n \mid n \in \mathbb{N} \} $
		\end{itemize}
	\end{block}
\end{frame}

\section{Rechtslineare Grammatiken}
\begin{frame}
	\frametitle{Motivation}
	\begin{itemize}
		\item Regulärer Ausdruck und endlicher Akzeptor sind äquivalent
		\pause
		\item Kontextfreie Grammatik ''kann mehr'' als regulärer Ausdruck und endlicher Akzeptor
		\pause
		\item Einführung einer eingeschränkten Version der Grammatik
	\end{itemize}
\end{frame}

\begin{frame}
	\begin{block}{Definition}
		\begin{itemize}
			\item $G = ( N,T,S,P)$ wie gehabt
			\item $\forall (w_1 \rightarrow w_2) \in P : (w_1 \in N) \land (w_2 \in \{ \epsilon \} \cup T \cup TN)$
			\pause
			\item Bei jeder Projektion steht rechts nur das leere Wort, ein Terminalsymbol oder ein Terminalsymbol gefolgt von einem einzigen Nichtterminalsymbol sein
		\end{itemize}
	\end{block}
\end{frame}

\begin {frame}
\frametitle {Ende}
	\begin {center}
		Noch Fragen?
	\end {center}
\end {frame}

\begin {frame}
\frametitle {Unnützes Wissen}
	\begin {center}
		Architektur war von 1912 bis 1948 eine olympische Disziplin
	\end {center}
\end {frame}

\end {document}