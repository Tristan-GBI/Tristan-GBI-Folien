\documentclass{beamer}
\usepackage[ngerman]{babel}
\usepackage[utf8]{inputenc}
\usepackage{graphicx}
\usepackage{amssymb} 
\usepackage{amsmath}

\usepackage{hyperref}

\usetheme{Warsaw}
\usecolortheme{default}


\author{Tristan Schnell}
\title{GBI-Tutorium 6}
\date{1.Dezember 2011}

\begin{document}

\begin {frame}
	\titlepage
\end {frame}

\begin {frame}
	\frametitle {Inhaltsverzeichnis}
	\tableofcontents
\end {frame}

\section{Wiederholung}
\subsection{Übungsblatt}

\begin{frame}
	\frametitle{Letztes \"Ubungsblatt}
	\begin{block}{Probleme}
		\begin{itemize}
			\item 5.1d) die Sprache mal genauer anschaun
			\item 5.2 $G=(N,T,S,P)$...
			\item 5.4 IA bei 0
			\item Teamarbeit
		\end{itemize}
	\end{block}
\end{frame}

\section{Übersetzung}
\subsection{Zahlendarstellung}
\frame {
	\frametitle{Zahlendarstellung}
	\begin{block}{Definition}
	Definiere num$_{10}$(x).\\
	\pause
	\hfill\\
	Sei Z$_{10}$ = $\{$0, 1, ..., 9$\}$, so definieren wir die Dezimaldarstellung von Zahlen so:\\
	\hfill\\
	Num$_{10}$($\epsilon$) = 0\\
	$\forall w \in$ Z$_{10}^*$ $ \forall x \in$ Z$_{10}$: Num$_{10}$(wx) = 10 $\cdot$ Num$_{10}$(x) + num$_{10}$(x)
	\end{block}
}
\frame {
	\frametitle{Andere Zahlendarstellungen}
	\begin{block}{Beispiele}
	Man kann nun nicht nur Zahle des im Zahlensystem der Basis 10 berechnen, sondern auch Zahler einer beliebigen Basis k.\\
	\hfill\\
	\begin{block}{Beispielaufgaben}
		\begin{itemize}
			\item Num$_{2}$(101) = \pause 5
			\item Num$_{5}$(431) = \pause 116
			\item Num$_{8}$(12) = \pause 10
		\end{itemize}
	\end{block}
	\end{block}
	
}

\subsection{Homomorphismen}
\frame {
	\frametitle{Übersetzungen}
	\begin{block}{Übersetzungen}
	Wozu braucht man überhaupt Übersetzungen?\\
	\pause
	\begin{itemize}
		\item Lesbarkeit
		\item Kompression
		\item Verschlüsselung
		\item Fehlererkennung und Fehlerkorrektur
	\end{itemize}
	\end{block}
}
\frame {
	\frametitle{Homomorphismen}
	\begin{block}{Präfixe}
	Präfixfreier Code: für keine zwei verschiedenen Symbole $x_1$, $x_2$ $\in$ A gilt: h($x_1$)
	ist ein Präfix von h($x_2$).\pause
	\\
	\hfill\\
	$\epsilon$-freier Homomorphismus\pause
	\\
	\hfill\\
	Homomorphismus: Seien A und B zwei Alphabete.\\
	$h: A \rightarrow B$ ist ein Homomorphismus, wenn gilt:\\
	\begin{center}
$h(\epsilon ) = \epsilon$\\
	$\forall w \in A^*: \forall x \in A: h(wx) = h(w)h(x)$
\end{center}
\end{block}
}

\subsection{Huffman-Code}
\frame {
	\frametitle{Huffman-Code}
	\begin{block}{Huffman}
	Der Huffman-Code ist ein Code, der unter allen präfixfreien Codes zu den kürzesten Codierungen führt.\\
	\hfill\\
	Wichtig ist dafür, dass wir die Anzahl gewisser Symbole unseres zu codierenden Textes kennen.\\
	\pause
	\begin{itemize}
		\item Für jedes zu kodierende Symbol erstellen wir einen Knoten, das das Symbol und seine Anzahl beinhaltet.
		\item Nun nehmen wir die zwei Knoten mit der kleinsten Anzahl, zählen die Anzahlen zusammen und erstellen einen Baum mit dem neu erstellten Knoten als Wurzel
		\item immersoweiter
		\item Wir beschriften alle Kanten, die nach rechts gehen mit 1 und alle nach links mit 0. 
	\end{itemize}
	\end{block}
}
\frame {
	\frametitle{Beispielaufgaben}
	\begin{block}{\bf Wir haben acht Symbole a, b, c, d, e, f, g, h}
		\begin{itemize}
			\item Jedes Zeichen kommt einfach vor. Wie sieht der Huffman-Code aus?\\
				Wie lang ist die Codierung von edcbahfg?\pause
			\item a kommt einmal vor, b zweimal, c 4-mal, d 8-mal, e 16-mal, f 32-mal,
g 64-mal, h 128-mal. Erstelle einen Huffman Baum.
		\end{itemize}	
	\end{block}
}
\frame {
	\frametitle{Block-Codierung}
	\begin{block}{Block-Codierung}
	Man kann natürlich nicht nur einzelne Symbole codieren, sondern auch Symbolblöcke.\\
	\hfill\\
	Wie würdet ihr den Huffman-Code für das folgende Wort definieren:\\
	$aaaaaabbbbbbccccccddddddaaadddddd$
	\end{block}
}
\section{Übungen}
\subsection{Übung 1}
\frame {
	\frametitle{Übung 1}
	\begin{block}{Klausur SS 2010}
	Gegeben sei dieser Baum. \pause
	\begin{itemize}
		\item Beschrifte die Kantes, sodass ein Huffman-Baum entsteht.
		\item Gib die Huffman Codierung des Wortes cae an.
		\item Gib paarweise verschiedene Häufigkeiten für a, b, c, d, e an, sodass sich bei der Huffman-Codierung obiger Baum entsteht.
	\end{itemize}
	\end{block}
}
\subsection{Übung 2}
\frame {
	\frametitle{Übung 2}
	\begin{block}{Klausur WS 2009/2010}
	 Gegeben sei das Alphabet A = $\{$a, b, c, d, e, f, g$\}$ und ein Wort w $\in$ A$^*$ in dem die Symbole mit den Häufigkeiten auf der Tafel vorkommen.
	 	\begin{itemize}
		\item Zeichne den Huffman-Baum 
		\item Gib die Huffman-Codierung für bad an \pause
		\item Fur k $\geq$ 1 sei ein Alphabet A = $\{$a$_0$, a$_1$, ... , a$_k$ $\}$ mit k + 1 Symbolen gegeben und ein Text, in dem jedes Symbol a$_i$ mit Häufigkeit 2$^i$ vorkommt für 0 $\geq$ i $\geq$ k.\\
Geben Sie die Huffman-Codierungen aller Symbole a$_i$ an.
	\end{itemize}
	\end{block}
}
\subsection{Übung 3}
\frame {
	\frametitle{Übung 3}
	\begin{block}{Klausur SS 2011}
	Gegeben sei ein Wort über der Alphabet A = $\{$a, b, c, d $\}$ mit den gegebenen relativen Häufigkeiten. \pause
	\begin{itemize}
		\item Erstelle den Huffman-Baum für x = $\frac{1}{16}$
		\item Für welche $x \in \mathbb{R}$ mit 0 $\geq x \geq \frac{1}{4}$ werden Wörter mit den angegebenen relaitven Häufigkeiten auf genau doppelt so lange Wörter über $\{$0,1$\}$ abgebildet?
	\end{itemize}
	\end{block}
}

\begin{frame}
	\begin{center}
		Fragen?
	\end{center}
\end{frame}

\begin{frame}
	\frametitle{Unnützes Wissen}
	\begin{center}
		Jack Nicholson fand erst mit 37 Jahren heraus, dass seine Schwester in Wahrheit seine Mutter ist.
	\end{center}
\end{frame}

\end {document}
