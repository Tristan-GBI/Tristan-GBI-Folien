\documentclass{beamer}
\usepackage[ngerman]{babel}
\usepackage[utf8]{inputenc}
\usepackage{graphicx}
\usepackage{amssymb} 
\usepackage{amsmath}

\usepackage{hyperref}

\usetheme{Warsaw}
\usecolortheme{default}


\author{Tristan Schnell}
\title{GBI-Tutorium 8}
\date{15.Dezember 2011}


\begin{document}

\begin {frame}
	\titlepage
\end {frame}

\begin {frame}
	\frametitle {Inhaltsverzeichnis}
	\tableofcontents
\end {frame}

\section{Wiederholung}

\begin{frame}
	\frametitle{Letztes \"Ubungsblatt}
		\begin{itemize}
			\item Auch mysteriöse Foren haben nicht immer recht
			\item Achtet auf Aufgabenstellung
		\end{itemize}
\end{frame}

\begin{frame}
	\frametitle{Wiederholung}
	\begin{center}
		$R^* = \bigcup_{k = 0}^\infty R^k$
	\end{center}
\end{frame}

\section{Algorithmen in Graphen} 
\subsection{Repr\"asentation von Graphen}

\begin{frame}
	\frametitle{Repr\"asentation von Graphen}

	\begin{block}{Adjazenzlisten}
		Sei G = (V, E) ein Graph, x $\in$ V ein Knoten. Die Adjazenzliste des Knotens x ist eine Liste $L_x \in V^{(*)}$, die alle 				Knoten y enth\"alt, f\"ur die \\ (x, y) $\in$ E (gerichteter Graph) bzw. \{x, y\} $\in$ E (ungerichteter Graph):
	\end{block}

	\begin{example}
		G = ( \{1, 2, 3\}, \{(1, 2), (1, 3), (2, 2), (3, 1), (3, 2), (3, 3) \} ) \\
		\parskip 12pt
		Darstellung mit Adjazenzliste: \\
		$L_1$ = (2, 3) \\
		$L_2$ = (2) \\
		$L_3$ = (1, 2, 3)\\
		
	\end{example}
\end{frame}


\begin{frame}
	\frametitle{Repr\"asentation von Graphen}
	\begin{block}{Adjazenzmatrix}
		Sei G = (V, E) ein gerichteter Graph und n := $\mid V \mid$. Eine Matrix A $\in$ $\{0, 1\}^{n x n}$ hei{\ss}t 					Adjazenzmatrix von G, wenn gilt: \\
		\begin{center}
			$\forall i, j \in V : A_{ij} = 
			\begin{cases}
				 1,  & \text{falls (i,j) $\in$ E}\\
				 0  & \text{sonst}
			\end{cases}$
		\end{center}

	\end{block}
\end{frame}

\begin{frame}
	\frametitle{Repräsentation von Graphen}
	\begin{example}
		G = ( \{1, 2, 3\}, \{(1, 2), (1, 3), (2, 2), (3, 1), (3, 2), (3, 3) \} ) \\
		\parskip 12pt
		Darstellung als Adjazenzmatrix:\\
		\begin{center}
		A = $
			\begin{pmatrix}
			  0 & 1 & 1\\
 			  0 & 1 & 0\\
			  1 & 1 & 1
			\end{pmatrix} $
		\end{center}
	\end{example}
\end{frame}

\begin{frame}
	\frametitle{Repr\"asentation von Graphen}
	\begin{block}{Vergleich von Adjazenzliste und -matrix}
	Adjazenzliste \\
	\begin{itemize}
		\item Die Frage ,,Welche Knoten sind zu x adjazent?" kann schnell beantwortet werden.	
		\item Spart bei geringer Kantenanzahl Speicherplatz.
	\end{itemize}
	Adjazenzmatrix \\
	\begin{itemize}
		\item Die Frage ,,Existiert eine Kante zwischen x und y?" kann schnell beantwortet werden.
		\item Ben\"otigt f\"ur feste Knotenanzahl immer gleich viel Speicher.
	\end{itemize}

	\end{block}
\end{frame}

\begin{frame}
	\frametitle{Repr\"asentation von Graphen}
	\begin{block}{Zu Adjazenzmatrizen}
	Woran erkennt man eine Schlinge? \\
	\uncover<2->{$\rightarrow$ 1 auf der Diagonalen. \\} 
	Welche Eigenschaften hat die Adjazenzmatrix eines ungerichteten Graphen?\\
	\uncover<3->{$\rightarrow$  Sie ist achsensymetrisch zur Hauptdiagonalen.\\}
	Welchen Graphen stellt folgende Matrix dar?\\
	\begin{center}
		A = 
		$\begin{pmatrix}
		 1 & 1 & 1 & 1\\
		 1 & 1 & 1 & 1\\
		 1 & 1 & 1 & 1\\
		 1 & 1 & 1 & 1
		\end{pmatrix}$
	\end{center}
	\end{block}
\end{frame}

\begin{frame}
	\frametitle{Repr\"asentation von Graphen}
	\begin{block}{Aufgabe}
		Gegeben sei der Graph G = (V, E) mit \\
		V = \{1, 2, 3, 4, 5, 6\} \\
		E = \{(1, 2), (1, 3), (3, 1), (3, 4), (3, 6), (4, 1), (5, 3), (5, 5), (6, 2), (6, 4), (6, 5)\} \\
		Gib die Adjazenzlisten und die Adjazenzmatrix an.\\
		Zeichne den Graphen.	
	\end{block}
\end{frame}



\begin{frame}
	\frametitle{Repr\"asentation von Graphen}
	\begin{block}{Aufgabe}
		Zeichne den Graphen mit der folgenden Adjazenzmatrix. \\
		\begin{center}
		A = 
		$\begin{pmatrix}
		 0 & 1 & 1 & 0\\
		 1 & 0 & 0 & 1\\
		 1 & 1 & 0 & 0\\
		 1 & 1 & 1 & 1
		\end{pmatrix}$
		\end{center}
	\end{block}
\end{frame}

\subsection{Wegematrix}

\begin{frame}
	\frametitle{Wegematrix}
	\begin{block}{Definition: Wegematrix }
		Sei G = (V, E) ein gerichteter Graph und n := $\mid V \mid$. Eine Matrix A $\in$ $\{0, 1\}^{n x n}$ hei{\ss}t 					Wegematrix von G, wenn gilt: \\
		\begin{center}
			$\forall i, j \in V : W_{ij} = 
			\begin{cases}
				 1,  & \text{falls es einen Pfad von i nach j gibt}\\
				 0  & \text{sonst}
			\end{cases}$
		\end{center}
	\end{block}
	\begin{example}
		Bestimme die Wegematrix.
	\end{example}
\end{frame}


\begin{frame}
	\frametitle{Wegematrix}
	\begin{block}{Zum Nachdenken}
		Wie sieht die Wegematrix aus, wenn die Adjazenzmatrix \\
		\begin{center}
			A = 
			$\begin{pmatrix}
				 1 & 1 & 1 & 1\\
				 1 & 1 & 1 & 1\\
				 1 & 1 & 1 & 1\\
				 1 & 1 & 1 & 1
			\end{pmatrix}$
		\end{center}
		ist?\\
		\uncover<3->{
 			Wann ist allgemein W=A?\\
		}
		\uncover<4->{
 		L\"osung: Wenn eine Kantenrelation E transitiv und reflexiv ist ($E^*$ = E)
		}
	\end{block}
\end{frame}

\subsection{Erreichbarkeitsrelationen}

\begin{frame}
	\frametitle{Erreichbarkeitsrelationen}
	\begin{block}{Definition: Erreichbarkeitsrelationen}
		Sei G =(V, E) ein Graph. Die reflexiv-transitive H\"ulle der Kantenrelationen E nennt man Erreichbarkeitsrelation $E^*$.				\\
		\begin{center}
		$E^* = \bigcup_{k = 0}^\infty E^k$
		\end{center}
	\end{block}

	\begin{block}{Definition: Wegematrix}
		Sei G = (V, E) ein gerichteter Graph und n := $\mid V \mid$. Eine Matrix \\
		W $\in$ $\{0, 1\}^{n x n}$ hei{\ss}t Wegematrix von G, wenn gilt: \\
		\begin{center}
			$\forall i, j \in V : W_{ij} = 
			\begin{cases}
				 1,  & \text{falls (i,j) $\in E^*$}\\
				 0  & \text{sonst}
			\end{cases}$
		\end{center}
	\end{block}
\end{frame}

\begin{frame}
	\frametitle{Matrixmultiplikation}
	Die Wegematrix l\"asst sich \"uber die Matrixmultiplikation errechnen.
	\begin{block} {Definition: Matritzenprodukt}
		Seien A eine (l x n)-Matrix und B eine (n x m)-Matrix. \\
		Dann hei{\ss}t die (l x m)-Matrix C mit:\\
		\begin{center}
			$C_{ij} = \sum_{k=0}^{n-1} A_{ik} \cdot B_{kj}$
		\end{center}
		das Produkt der Matritzen A und B.\\
		Man schreibt C = A $\cdot$ B
	\end{block}
\end{frame}

\begin{frame}
	\frametitle{Potenzen der Adjazenzmatrix}
	\begin{block}{Aufgabe}
		Sei A die Adjazenzmatrix des Graphen G = (V, E). Berechne die Wegematrix W mit Hilfe der Matrixmultiplikation. \\
				
			\begin{center}
			A =	
			$\begin{pmatrix}
				 0 & 1 & 1 & 0\\
				 0 & 1 & 0 & 1\\
				 1 & 1 & 1 & 0\\
				 0 & 1 & 0 & 0
			\end{pmatrix}
			=A^1$
		\end{center}
	

	
		
	\end{block}
\end{frame}

\begin{frame}
	\frametitle{L\"osung}
	\begin{block}{L\"osung}
	
	
	$A^0 = \begin{pmatrix}
				 1 & 0 & 0 & 0\\
				 0 & 1 & 0 & 0\\
				 0 & 0 & 1 & 0\\
				 0 & 0 & 0 & 1
			\end{pmatrix}$
	$A^2 = \begin{pmatrix}
				 1 & 2 & 1 & 1\\
				 0 & 2 & 0 & 1\\
				 1 & 3 & 2 & 1\\
				 0 & 1 & 0 & 1
			\end{pmatrix}$
	$A^3 = \begin{pmatrix}
				 1 & 5 & 2 & 2\\
				 0 & 3 & 0 & 2\\
				 1 & 7 & 3 & 3\\
				 0 & 2 & 0 & 1
			\end{pmatrix}$  

	$\sum_{k=0}^{n-1} A^k = 
			\begin{pmatrix}
				 3 & 8 & 4 & 3\\
				 0 & 7 & 0 & 4\\
				 4 & 11 & 7 & 4\\
				 0 & 4 & 0 & 3
			\end{pmatrix}$
	$W = sgn(\sum_{k=0}^{n-1} A^k) =
			\begin{pmatrix}
				 1 & 1 & 1 & 1\\
				 0 & 1 & 0 & 1\\
				 1 & 1 & 1 & 1\\
				 0 & 1 & 0 & 1
			\end{pmatrix}$

	\end{block}	
\end{frame}

\begin{frame}
	\frametitle{Algorithmus zum Berechnen von $E^*$}
	\begin{block}{Code}
		W $\leftarrow$ 0\\
		\textbf{for} i $\leftarrow$ 0 \textbf{to} n-1 \textbf{do}\\
		\hspace{0.5 cm}M $\leftarrow$ Id\\
		\hspace{0.5 cm}\textbf{for} j$\leftarrow$ 1 \textbf{to} i \textbf{do}\\
		\hspace{1 cm}M $\leftarrow$ M $\cdot$ A\\
		\hspace{0.5 cm}\textbf{od}\\
		\hspace{0.5 cm}W $\leftarrow$ W + M\\
		\textbf{od} \\
		W $\leftarrow$ sgn(W)
	\end{block}
	\begin{block}{Berechnungszeit}
	Dieser Code benötigt ungef\"ahr $n^5$ Arithmetische Operationen (AO) zum berechnen der Wegematrix
	\end{block}
\end{frame}

\subsection{Rechenaufgaben}

\begin{frame}
	\frametitle{Rechenaufgabe}
	\begin{block}{Ein schneller Algorithmus}
		Angenommen, man hat eine feste Gesamtrechenzeit t zur Verf\"ugung.\\
		Der Algorithmus, der $n^5$ AO ben\"otigt (Alg0), k\"onnte in dieser Zeit \\
		$k_0$ = 1000 Probleminstanzen abarbeiten (Wegematritzen f\"ur 1000 Graphen \\
		mit n Knoten erzeugen).\\
		\hspace{0.5 cm} Wie viele Probleminstanzen $k_1$ w\"urde der Algorithmus, der nur $n^4$ AO \\
		ben\"otigt, in der gleichen Zeit schaffen?
	\end{block}
\end{frame}

\begin{frame}
	\frametitle{Rechenaufgabe}
	\begin{block}{L\"osung}
	Alg0 bearbeitet in der Zeit t $k_0$ mal $n^5$ AO: t = $k_0 \cdot n^5$\\
	Alg1 bearbeitet in der Zeit t $k_1$ mal $n^4$ AO: t = $ k_1 \cdot n^4$ \\
	Damit gilt: \\
	\hspace{5 cm} $k_0 \cdot n^5 = k_1 \cdot n^4 $ \\
	\hspace{5 cm} $\Leftrightarrow k_1 = \frac{k_0 \cdot n^5}{n^4} $\\
	\hspace{5 cm} $\Leftrightarrow k_1 = k_0 \cdot n$\\
	\hspace{5 cm} $\Leftrightarrow k_1 = 1000 \cdot n$

	\end{block}
\end{frame}

\subsection{Algorithmus von Warshall}


\begin{frame}
	\frametitle{Algorithmus von Warshall}

	\hspace{0.5 cm} \textbf{for} i $\leftarrow$ 0 \textbf{to} n - 1 \textbf{do} \\
	\hspace{1 cm} \textbf{for} j $\leftarrow$ 0 \textbf{to} n - 1 \textbf{do} \\
	\hspace{1.5 cm}	$W_{ij} \leftarrow 
			\begin{cases}
				1,  & \text{falls i = j}\\
				 A_{ij}  & \text{falls i $\ne$ j}
			\end{cases}$ \\
	\hspace{1 cm} \textbf{od}\\
	\hspace{0.5 cm} \textbf{od}\\
	\vspace{0.5 cm}
	\hspace{0.5 cm} \textbf{for} k $\leftarrow$ 0 \textbf{to} n - 1 \textbf{do} \\
	\hspace{1 cm} \textbf{for} i $\leftarrow$ 0 \textbf{to} n - 1 \textbf{do} \\
	\hspace{1.5 cm} \textbf{for}j $\leftarrow$ 0 \textbf{to} n - 1 \textbf{do} \\
	\hspace{2 cm} $W_{ij} \leftarrow max( W_{ij}, min(W_{ik}, W_{kj}) )$\\
	\hspace{1.5 cm} \textbf{od}	\\
	\hspace{1 cm} \textbf{od}	\\
	\hspace{0.5 cm} \textbf{od}	
\end{frame}

\begin{frame}
	\frametitle{Algorithmus von Warshall}
	\begin{block}{Aufgabe}
	Gib zu folgenden Graphen jeweils die Adjazenzmatrix an und bestimme die 
	reflexiv-transitive H\"ulle mit Hilfe des Warshall-Algorithmus. Kennzeiche 
	bei jedem Schritt des Algorithmus die Ver\"anderungen.
	\end{block}
\end{frame}


\begin {frame}
\frametitle {Ende}
	\begin {center}
		Fragen?
	\end {center}
\end {frame}

\begin {frame}
\frametitle {Unnützes Wissen}
	\begin {center}
		In der Schweiz gibt es bei manchen Geschirrspülmaschinen extra ein Käsefondue- und Racletteprogramm.	
	\end {center}
\end {frame}

\end {document}
