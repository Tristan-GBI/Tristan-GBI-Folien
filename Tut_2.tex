\documentclass{beamer}
\usepackage[ngerman]{babel}
\usepackage[utf8]{inputenc}
\usepackage{graphicx}
\usepackage{amssymb} 

\usetheme{Warsaw}
\usecolortheme{default}


\author{Tristan Schnell}
\title{GBI-Tutorium 2}
\date{3.November 2011}


\begin{document}

\begin {frame}
	\titlepage
\end {frame}

\begin {frame}
	\frametitle {Inhaltsverzeichnis}
	\tableofcontents
\end {frame}

\section{Wiederholung}

\begin{frame}
	\frametitle{Letztes \"Ubungsblatt}
	\begin{block}{Aufgabe 1.2}
		Tafel
	\end{block}

	\begin{block}{Aufgabe 1.3}
	        	\begin{itemize}
			\item M kann unendlich sein!
			\item einfache Lösung: Gegenbeispiel
		\end{itemize}
	\end{block}

	\begin{block}{Aufgabe 1.4a}
		\begin{itemize}
			\item oft richtige Antwort
			\item Begründungen teilweise seeehr fragwürdig
		\end{itemize}
	\end{block}
\end{frame}

\begin {frame}
	\frametitle {Wiederholung}
	\begin{block}{Alphabet}
		Ein Alphabet ist eine:
		\pause
        		\begin{itemize}
			\item endliche
			\item nicht leere
			\item Menge von Zeichen
		\end{itemize}
	\end{block}
\end {frame}

\section{Prädikatenlogik} 
\frame {
	\frametitle{Prädikatenlogik}
	Mit der Prädikatenlogik können wir viele Sachverhalte kurz und präzise darstellen.
	\pause
	
	\begin{block}{}
	Sei $W$ die Menge der möglichen Wetterformen und $S$ die Menge aller Studenten.\\
	$\heartsuit \subseteq$ $S \times W$ beschreibt "`Student liebt das Wetter"'
	\pause
	\begin{itemize}
		\item $\neg\exists s \in S$ : $\forall w \in W$ : $s \heartsuit w$\\
			\pause
			Es existiert kein Student, der alle Wetterformen liebt.
			\pause
		\item $\exists w \in W$ : $\forall s \in S$ : $s \heartsuit w$\\
			\pause
			Es existiert eine Wetterform, die jeder Student liebt.
			\pause
		\item $\forall s \in S$ : $\exists w \in W$ : $s \heartsuit w$\\
			\pause
			Für alle Studenten existiert eine Wetterform, die er liebt.
	\end{itemize}
	\end{block}
}

\section{W\"orter}
\begin{frame}
	\frametitle{Wörter}
	\begin{block}{Vorbemerkung}
		\begin{itemize}
			\item $\mathbb G_n$  = \pause \{ i $ \in $ $\mathbb N_0$ $ | $ 0 $ \le $ i $ \land $ i $ < $ n \} 
			\item $\mathbb G_0$ = \pause \{\}
		\end{itemize}
	\end{block}
	\begin{block}{In Worten}
		Wörter sind eine Surjektive Abbildung mit w: $\mathbb G_n$  $\rightarrow$ B \subset A \\
		\begin{example}
		Das Wort w = hallo ist eine Abbildung  \\ 
		 w: $\mathbb G_5$ $\rightarrow$ \{ a,h,l,o \} mit \\
		 w(0) = h w(1) = a w(2) = l w(3) = l w(4) = o
		\end{example}
	\end{block}
\end{frame}

\subsection{Das leere Wort}
\begin{frame}
	\frametitle{Das leere Wort}
	\begin{block}{Das Wort}
		\begin{itemize}
			\item Das leere Wort wird mit dem $\epsilon$ dargestellt, und ist eine \\ 
				Abbildung von \{\} $\rightarrow$ \{\} \\
			\item \{\} x \{\} = \{\} \\
			\item $\epsilon$ hat die Länge 0 ist aber dennoch ein Element. \\
			\item wenn M = \{$\epsilon$\} dann ist M $\not=$ $\emptyset$ \\
			\item $|$M$|$ = 1
		\end{itemize}
	\end{block}
\end{frame}

\subsection{Konkatenation}
\begin{frame}
	\frametitle{Konkatenation von Wörtern}
	\begin{block}{Konkatenation von Wörtern}
		\begin{itemize}
			\item eine Konkatenation ist eine Verknüpfung mehrerer Zeichen(ketten) und wird als $\cdot$ dargestellt \\
			\item z.B. kann man hallo als h $\cdot$ a $\cdot$ l $\cdot$ l $\cdot$ o dargestellt werden.\\
			\item der Punkt ist allerding nicht notwendig, er kann wie das Malzeichen bei der Multiplikation weggelassen werden. \\
			\item mehrere Wörter können auch zu einem weiteren konkateniert werden.
		\end{itemize}
	\end{block}
\end{frame}

\section{Vollst\"andige Induktion}
\subsection[Einf\"uhrung]{Einf\"uhrung}
	
\begin{frame}
	\frametitle{Ein beliebiger \"Uberblick}
	\begin{block}{Was ist die vollst\"andige Induktion?}
		Eine oft benutzte sehr m\"achtige Beweistechnik
	\end{block}
	\begin{block}{Vorgehen?}
		\begin{enumerate}
  			\item Die Behauptung f\"ur einen ersten Wert beweisen
		 	\item Annehmen dass die Behauptung f\"ur ``irgendeinen'' Wert gilt
  			\item Behauptung ausgehend von dem bliebigen Wert f\"ur den n\"achsten Wert
  			beweisen
		\end{enumerate}
	\end{block}
\end{frame}

\begin{frame}
	\frametitle{So sollte es aussehen}
	\begin{block}{Induktionsanfang}
		\begin{itemize}
 			\item Beweis der Behauptung für einen (manchmal auch mehrere) ''Startwerte''
		\end{itemize}
	\end{block}
	\pause
	\begin{block}{Induktionsannahme}
		\begin{itemize}
  			\item F\"ur ein beliebiges aber festes x/k/n gelte: \ldots
  			\item Wird im Induktionsschritt benutzt
		\end{itemize}
	\end{block}
	\pause
	\begin{block}{Induktionsschritt}
		\begin{itemize}
  			\item Ausgehend von x die Behauptung f\"ur $x + 1$ beweisen
		\end{itemize}
	\end{block}
\end{frame}

\begin{frame}
	\frametitle{Ein erstes Beispiel}
	\begin{block}{Die Gaußsche Summenformel}
		\begin{center}
			$\sum\limits_{k=1}^{n}k  =  1 + 2 + 3 + 4 + \ldots + n =
			\frac{n(n+1)}{2}$
		\end{center}
	\end{block}
\end{frame}

\begin{frame}
	\frametitle{Beweis!}
	\begin{block}{Induktionsanfang}
		\begin{center}
			$n = 1:$ \hspace{10mm}  $\sum\limits_{k=1}^{n}k$ = $\sum\limits_{k=1}^{1}k$ =  1
			= $\frac{1(1+1)}{2}$ = $\frac{n(n+1)}{2}$ \hspace{15mm} \uncover<2->{$\Box$}
		\end{center}
	\end{block}

	\uncover <3-> {
	\begin{block}{Induktionsvorraussetzung}
		\begin{center}
			 F\"ur ein beliebiges aber festes n gelte: \\
			$\sum\limits_{k=1}^{n}k = \frac{n(n+1)}{2}$
		\end{center}
	\end{block} }

	\uncover <4-> {
	\begin{block}{Induktionsschluss}
		\begin{center}	
		$n = 1:$ \hspace{10mm} 
		$\sum\limits_{k=1}^{n+1}k = (n+1)+ \sum\limits_{k=1}^{n}k \stackrel{\mathrm{I.V.}}= 
		(n+1)+\frac{n(n+1)}{2} $ \\
		$= \frac{(n+1)(n+2)}{2}  $ \uncover<5->{$\Box$}
		
		\end{center}
	\end{block} }
\end{frame}

\subsection[Aufgaben]{Aufgaben}
\begin{frame}
	\frametitle{Jetzt seid ihr dran}
	\begin{block}{Eine Reihe}
		\begin{itemize}
  			\item $a_{0} = 0$
  			\item $a_{n+1} = a_{n} + 2n + 1$
		\end{itemize}
	\end{block}
	
	\vspace{5mm}

	\uncover<2-> 
	{
 	\begin{block}{Zeige}
	$a_n = n^2$
	\end{block} 
	}
\end{frame}

\begin{frame}
\frametitle{Weiter gehts}
	\begin{block}{Noch ne Reihe}
		\begin{itemize}
  			\item $a_{0} = 3$
  			\item $a_{n+1} = a_{n} + 3$
		\end{itemize}
	\end{block}

	\vspace{5mm}
	\uncover<2->
	{
	\begin{block}{Zeige}
		\begin{itemize}
  			\item Ideen?
  			\visible<3->{\item $a_{n} = 3(n+1)$}
		\end{itemize}
	\end{block}
	}
\end{frame}

\begin{frame}
\frametitle{Und jetzt mal was schweres}
	\begin{block}{Aufgabe }
		\begin{itemize}
			\item $x_0$ = 0 \\
			\item $\forall n \, \in \mathbb N_0 : x_{n+1} = x_n + (n+1)(n+2)$
			\item Tipp: $x_1, x_2, x_3, x_4$ ausrechnen
			\visible<2->{\item Wenn keine Idee: $ \frac{x(x+1)(x+2)}{3}$}
		\end{itemize}
	\end{block}
\end{frame}

\begin {frame}
\frametitle {Ende}
	\begin {center}
	Fragen?! \\
	\end {center}
\end {frame}

\begin{frame}
\frametitle{Unnützes Wissen}
	\begin{center}		
		Ein Liter Druckertinte von Hewlett Packard kostet mehr als ein Liter Chanel No. 5
	\end{center}
\end{frame}

\end {document}
